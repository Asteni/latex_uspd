\documentclass[a4paper,12pt]{article}
\usepackage[T2A]{fontenc}
\usepackage[utf8]{inputenc}
\usepackage[russian]{babel}
\usepackage{uspd}
\setstudent{Васильев~Б.~А}
\setgroup{БПИ173}
\setdocumenttype{Техническое задание}
\setprofname{Полицына~Е.~В.}
\setproftitle{Научный руководитель, доцент департамента программной инженерии факультета компьютерных наук, канд. техн. наук}
\setcode{RU.17701729.507600-01 90 01}

\begin{document}
\sloppy
\title{ПРОГРАММНЫЕ СРЕДСТВА ДИНАМИЧЕСКОГО
РАСПАРАЛЛЕЛИВАНИЯ ПРОГРАММ: Т--СИСТЕМА \\
}

\uspdabstract{Целью данного документа является ознакомление
системного программиста с архитектурой программной среды Т--системы.}
\begin{uspd}{\code}

\input{ts.tex}

\end{uspd}

\end{document}
